\documentclass{book}

% listings package is used to print out source code with line numbers
\usepackage{listings}

% color package allows us to define colors for --color option
\usepackage[usenames,dvipsnames]{color}

% paratype package is provided by the OpenSUSE/fedora package texlive-paratype
\usepackage{paratype}

% fancyhdr simplifies page headers
\usepackage{fancyhdr}

% Enable fancyhdr
\pagestyle{fancy}

% all even pages have title (\leftmark) and page numbers on the outside
\fancyhf{}
\fancyhead[RE]{\leftmark}
\fancyhead[LE,RO]{\thepage}

% Use monospace font for everything by default
\renewcommand{\familydefault}{\ttdefault}

% Define custom colors for highlighting, if using color printing
\definecolor{OliveGreen}{cmyk}{0.64,0,0.95,0.40}
\definecolor{gray}{rgb}{0.5, 0.5, 0.5}
\definecolor{black}{rgb}{0.0, 0.0, 0.0}

% Define the title and table of contents
\title{\VAR{title}}
\date{\VAR{releasedate}}

% Display title on left-side pages and above table of contents
\renewcommand*\contentsname{\VAR{title}}


\begin{document}
% Title page with no header/footer
\thispagestyle{empty}
\maketitle
\newpage

% Blank left page
\thispagestyle{empty}


% Begin table of contents
\cleardoublepage
\thispagestyle{empty}
\addcontentsline{toc}{chapter}{ls -l \VAR{contentsdir}}
\tableofcontents
\cleardoublepage

% Include the GPL v2 license
\fancyhead[LO]{LICENSE}
\include{gplv2}

% Now loop through all sections to include
\BLOCK{ for section in sections }

\fancyhead[LO]{\VAR{section.title}}
\include{\VAR{section.tex_file}}

\BLOCK{ endfor }

\end{document}
